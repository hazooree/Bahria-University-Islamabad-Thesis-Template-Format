\newpage
\addcontentsline{toc}{chapter}{Abstract}
\newgeometry{left=3.5cm,right=2.5cm,top=4.0cm,bottom=3.5cm}
\begin{center}
    \large \textbf{Abstract}
\end{center}
\vskip 0.45in


The purpose of this study is to investigate the application of genetic
algorithm (GA) in modelling linear and non-linear dynamic systems and develop an
alternative model structure selection algorithm based on GA. Orthogonal least square
(OLS), a gradient descent method was used as the benchmark for the proposed
algorithm. A model structure selection based on modified genetic algorithm (MGA)
has been proposed in this study to reduce problems of premature convergence in
simple GA (SGA). The effect of different combinations of MGA operators on the
performance of the developed model was studied and the effectiveness and shortcomings
of MGA were highlighted. Results were compared between SGA, MGA and benchmark
OLS method. It was discovered that with similar number of dynamic terms, in most
cases, MGA performs better than SGA in terms of exploring potential solution and
outperformed the OLS algorithm in terms of selected number of terms and predictive
accuracy. In addition, the use of local search with MGA for fine-tuning the algorithm
was also proposed and investigated, named as memetic algorithm (MA). Simulation
results demonstrated that in most cases, MA is able to produce an adequate and
parsimonious model that can satisfy the model validation tests with significant
advantages over OLS, SGA and MGA methods. Furthermore, the case studies on
identification of multivariable systems based on real experimental data from two systems
namely a turbo alternator and a continuous stirred tank reactor showed that the proposed
algorithm could be used as an alternative to adequately identify adequate and
parsimonious models for those systems.
\newpage
\newgeometry{left=3.5cm,right=3.5cm,top=4.0cm,bottom=3.5cm}
